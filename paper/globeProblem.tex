New networking protocols hardly ever gain traction. Businesses are not likely to use a new protocol if its benefits depend on whether both the host and the receiver have implemented the new protocol, especially if the protocol requires new infrastructure. Host Identity Protocol (HIP) suffers from this problem, leading to limited use of the protocol. ``HIP secures network data flows of applications, works well with IPv6 and NAT traversal, and allows seamless switching between difference networks, which keeps data streams from being disrupted during the transition'' \cite{Leva:2013:ABN:2494664.2494695}. 

None of its features are completely new, but it is the only protocol that supports all those features in one package. Even though many businesses may have benefited from the widespread adoption of HIP (especially in mobile communications), HIP was not able to make it past several initial barriers to adoption. Businesses could get similar functionality from using several different protocols that were already in widespread use, and HIP needed both the sender and receiver to support HIP in order for it to actually have any benefits over other protocols. 

Though HIP had a more complete feature set that could of potentially benefited businesses more than other protocols, Hip was unable to compete against existing protocols that were more useful because they were more widely adopted, a phenomenon known as the  ``network effect''  \cite{Leva:2013:ABN:2494664.2494695}.  This ``network effect'' is the reason most new networking protocols fail to gain traction. 

This often is a good thing because switching protocols is a big change, and can cause fragmentation of standards. But there is a downside: if a better protocol is developed, it is often ignored because its capabilities cannot be realized until enough people adopt it. There are however, some ways of deploying a new protocol where the lack of widespread adoption doesn’t matter. 

HIP is used by Boeing to ``secure traffic (and identify machines) with moving robots at their airplane factory'' 
\cite{Leva:2013:ABN:2494664.2494695}.  The features of HIP were able to be utilized in a relatively small setting even though it was not a popular protocol. New protocols can use these very specific implementations like this one to showcases what make makes it valuable without it needing universal conformity.