Standards are an important in all parts of networking.  They are especially important when it comes to network security protocols.  Standards in security have both an upside and a downside.  One of the main upsides is that standard protocols allow for virtually any business that is looking to create a product using network features to secure them without doing much research or development on their own.  Utilizing a standard security protocol also means that if you are creating a product for the general public your device will be able to interface more easily with those created by other businesses.  Standardized protocols also have the benefit of being well tested because they are used in so many places.  The fact that they are widely used means that more work goes into making sure they are secure.  This benefit also has a flip side that produces one of the major drawbacks from network security protocols being standardized. The ubiquity of these standard security protocols also means that they are widely known to the general public, not just to the businesses using them.  This means that the risk of a widely known protocol being broken and the security of systems using that protocol at risk is much higher than a security protocol a company develops on their own.  A standard protocol therefore is always at risk of being compromised.  

One such set of network security protocols are those for wireless networks.  The two main protocols for wireless connectivity are WEP (Wired Equivalent Privacy) and WPA (Wi-Fi Protected Access).  WEP was the original wireless security standard when Wi-Fi first gained popularity.  Multiple security issues were found with WEP due to the fact that WEP used static shared secrets keys for long periods of time.  This left networks vulnerable once the secret keys were decrypted and showed signs that improvements needed to be made to the protocol\cite{5711033}. WEP2 was introduced as a potential successor to WEP and tried to fix the security problems with the original standard.  WEP2 however proved to be deficient as well and it was concluded that the entire WEP algorithm was not sufficient enough to stand up as the standard wireless network security protocol\cite{5234856}. WPA was then introduced as the successor to WEP.  WPA became the standard wireless networking protocol because it solved many of the cryptographic issues that WEP had.  WPA uses a more complex encryption algorithm known as TKIP (Temporal Key Integrity protocol)\cite{5234856}.  WPA also includes a message integrity check, a per-packet mixing function, and a re-keying mechanism that all address specific problems that WEP had.  In the case of wireless network security protocol standards we see that a standard was made and widely used, but ran into security issues because it had become so known.  As a response to this the standard evolved and was able to improve upon the specific problems that the last iteration of the standard had run into.  This shows how that while having a network standard can be hazardous to security it can also help to refine and develop the protocols, as long as the market and businesses are willing to adapt and push the standard forward.  
