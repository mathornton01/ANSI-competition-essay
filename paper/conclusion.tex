\section{Conclusion}
In this paper we examined three specific kinds problems that relate to Network Protocols and the effects that they have on persons wishing to enter the market place. We first examined how the new business owner is faced with an initial trade off in choosing a protocol, in terms of breadth vs. sustainability. We used the example of the increasing address space and Skype to exemplify how a companies future, and initial success is almost completely reliant upon choosing the right protocol to use. We next focused our efforts on the influx of security protocols that have made an appearance in the past twenty years. We examined the evolution of the wifi security protocols WEP and WPA. We concluded that these policies can contribute to an early stage company in several good ways if they are used correctly. For example, a company owner who purchases bandwidth from his local area provider can use these Wi-Fi security protocols in order to protect their asset from theft. However if used incorrectly (I.e. using a bad password) these protocols are almost certainly only going to result in user frustration and lost work-hours. Finally we looked at protocols that aren't widely adapted, and vary in adaption rate largely from region to region. We specifically focus on HIP. We note that the variance in adoption rate among different geographical regions can pose both economic, and technical problems to the early entrepreneur.


In conclusion we would like to note that it would be completely impossible to conduct business of any kind without protocols. We even use protocols when we speak to one another. There is no avoiding them, the focus of our paper was just on how they might cause difficulties for an early business start up. 